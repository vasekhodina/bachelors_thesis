\documentclass[color,table,oneside,nolot,nolof]{fithesis}
\usepackage[resetfonts]{cmap}
\usepackage[main=czech,english]{babel}
\usepackage[utf8]{inputenc}
\thesissetup{
		university = mu,
		faculty = fi,
		type = bc,
		author = Václav Hodina,
		gender = m,
		advisor = Marek Grác,
		title = {Vizualizace rozdělování disků},
		keywords = {vizualizace, instalace, rozdělení disků, linux, LVM, RAID},
}
\thesislong{abstract}{
	Práce popisuje vývoj modulu do instalátoru Anaconda, který využívají linuxové distribuce vycházející z~Red Hat Enterprise Linuxu.
}
\thesislong{thanks}{
	Zde chci poděkovat Marku Grácovi za vedení mé práce, Marii Staré za korektury.
}
\usepackage{csquotes}
\usepackage[plainpages=false,pdfpagelabels,unicode]{hyperref}
\usepackage{charter,graphicx}
\usepackage[top=1in, bottom=1.25in, left=1.25in, right=1.25in]{geometry}
\usepackage[
	backend=biber,
	style=numeric,
	citestyle=numeric-comp,
	sorting=none,
	sortlocale=auto
	]{biblatex}
\addbibresource{bak_prace.bib}
\usepackage{makeidx}
\makeindex
\usepackage{paralist} 
\usepackage{amsmath} 
\usepackage{amsthm} 
\usepackage{amsfonts} 
\usepackage{url} 
\usepackage{menukeys}
\hyphenation{graph-viz}
\begin{document}
\chapter{Úvod}
	Bakalářská práce zpracovává řešení problémů s~vizualizací rozdělených disků při instalaci systému. Cílem práce je vytvořit pochopitelnou grafickou nápovědu pro administrátory 
	počítačů, zvláště serverů s~mnoha disky. 
	Konkrétněji se jedná o~rozšíření instalátoru Anaconda, které zpracovává informace o~jednotlivých discích, jako je název, velikost a typ disku a~jednotlivé oddíly na disku utvořené. 
	Samozřejmostí je zahrnutí 
	diskových polí typu RAID (Redundant Array of Independent Disks)  a~virtualizovaných disků mezi vizualizovaná data; rozšíření počítá se všemi těmito typy. Data jsou uložena 
	ve vlastních třídách tak, aby programování případné další funcionality 
	nepředstavovalo problém. Program vytváří graf podobný stromové struktuře a~zobrazuje jej uživateli. Graf se během instalace tvoří dvakrát, poprvé
	před rozdělením disků a~podruhé pro kontrolu, zda jsou předložené změny korektní, než se zformátují disky. V~současnosti je k~tomuto účelu využíván pouze textový 
	seznam změn, který je nedostatečný. Člověk dokáže mnohem lépe a~rychleji kontrolovat obrázková data než homogenní text. 

	Práce vznikala nejen na Fakultě informatiky Masarykovy univerzity (FI MUNI), ale i~ve společnosti Red Hat. Tam budou využity její výsledky,
	integrované do instalátoru Anaconda, který je v~současnosti používán v~linuxových distribucích Red Hat Enterprise Linux (RHEL), CentOs a~Fedora.

	Práci jsem si vybral z~několika důvodů.  Možnost podílet se na vývoji svobodného softwaru je pro mě velmi důležitým hlediskem 
	při psaní jakéhokoliv programu. Druhý důvod je možný rozsah uplatnitelnosti výsledků mé práce. Každý systém je třeba nejprve nainstalovat, výsledky
	této práce tedy uvidí velké množství lidí, což je bezpochyby velká motivace pro každého, kdo něco tvoří. Třetím důvodem je výběr programovacího jazyka, který je vyžadován 
	v~zadání. Jedná se o jazyk Python, který považuji za velmi flexibilní, aniž by byly kladeny přílišné nároky na výkon systému. 

	Jak jsem zmínil výše, hlavním cílem práce je naprogramování aplikace, která vytváří graf stromové struktury rozdělených disků. Data přijímá od instalátoru Anaconda, 
	přičemž využívá knihovnu blivet. 
	Vedlejšími cíli je možnost vytvářet grafy ze souborů XML, ne jen z~aplikace, dále integrace do instalátoru s~možností označovat jednotlivé diskové oddíly a~interagovat
	s~nimi. Posledním doplňkovým cílem je funkcionalita umožňující porovnat stav před instalací a~po ní v~případě, že je systém přeinstalováván.

	Z~cílů vychází také struktura práce. První kapitola popisuje použité knihovny. První knihovna, blivet\cite{blivet}, poskytuje data a mimo jiné může sloužit i~pro změny 
	nastavení disků (tato funkcionalita je v~mé práci zmíněna jen okrajově). Druhou je knihovna graphviz-python\cite{graphviz-python}, Graphviz je program pro tvorbu grafů. Kromě
	jednoduchého spojování uzlů hranami dokáže uzly automaticky třídit a~logicky rozmisťovat podle různých přednastavených pravidel. Nabízí také různý vzhled uzlů
	a~hran a výsledné grafy dokáže exportovat v~několika formátech. Výsledek mé práce operuje především s~formátem škálovatelné vektorové grafiky (SVG). Graphviz-python je 
	nadstavba Graphvizu pro použití v~jazyce Python.

	Druhá kapitola je o~mém návrhu jednotlivých tříd programu, jejich dokumentaci a~popisu funkcí. Nejdůležitější z tříd jsou ty pro uzly a~hrany. Zmíněny jsou také pomocné třídy 
	pro načítání z~jiných formátů vstupních dat, jako je již zmiňované XML.

	Třetí kapitola obdobně popisuje návrh vzhledu aplikace a~její chování. Zdůvodňuje, proč jsem se rozhodl pro jednotlivé grafické prvky a~barevná odlišení.

	Čtvrtá kapitola obsahuje ukázky práce programu. Demonstruje několik konfigurací, jež mají za úkol program otestovat a~vyzkoušet i~potenciálně problémové situace. Ukázky  
	zahrnují situace jak při práci v~prostředí instalátoru, tak mimo něj.

	Pátá kapitola zmiňuje další možná rozšíření mého programu. 

\chapter{Stať}
\section{Knihovna Blivet}
	První a nejdůležitější knihovouk která je v mé práci využívána je knihovna blivet. Jde o projekt a bakalářsknou práci Bc. Vojtěcha Trefného. Projekt vznikl, stejně jako moje
	práce ve firmě Red Hat a slouží k rozšíření již zmiňovaného instalátoru Anaconda. Použití této knihovny je součástí zadání a proto nebudu diskutovat její výhody a nevýhody
	oproti ostatním knihovnám. 

	Blivet však není knihovnou která by jen četla informace o pevných discích. Mezi její funkce patří i konfigurace různých datových úložišť. Nemusí se ani jednat jen o pevné disky.
	Ovládá i mnohé další technologie se kterými se lze v dnešní době setkat. Příkladem jsou to vícenásobné pole nezávislých disků (RAID), technologie logických svazků disku (LVM) či 
	ovládání zašifrovaných modulů pomocí technologie LUKS. Všechny 3 příklady proberu níže.

	Dále obsahuje též nástroje pro práci se souborovými systémy diskových oddílů od starších a již používaných jako jsou žurnálovací souborové systémy ext2 až ex4 či ReiserFS po novější
	jakými jsou btrfs nebo ZFS. Též se stará o bootovací oddíly čili master boot record (MBR) a GUID partition table (GPT). Stručně řečeno stará se o všechny součásti procesu instalace
	nového systému na počítač.

\subsection{RAID}
	Vícenásobná pole nezávislých disků jsou velmi elegantní ochranou před selháním disků. Existůjí různé způsoby jak pole realizovat ale základní princip zůstává vždy stejný. Jde v~něm
	o~několik disků, které nakonec vystupují jako jeden disk. Podle použítí typu vícenásobného diskového pole může mít tento disk kapacitu rovnou součtu disků jej tvořících, anebo také
	jen kapacitu jednoho disku, přičemž data jsou zrcadlena na ostatní disky. Cílem tohoto nastavení je ochrana před selháním hardware a ztrátou dat. 
	
	Blivet obsahuje nástroje pro práci s svobodnou technologií mdadm která slouží k nastavení softwarového RAID pole. Při využití hardwarových technologií, zvláště pak nesvobodných
	je možné k tomuto RAIDu přistupovat jako k obyčejnému disku, čehož i nesvobodné RAIDy často využívájí a svých mnoho disků schovávají za jednotným rozhraním, které se tváří jako jeden
	disk. 

	Program mdadm je ale softwarový RAID a proto má počítač celou dobu přehled nejen o finálním zabezpečeném disku proti selhání ale i o všech dílčích discích které ho tvoří. Výhodou je
	možnost monitoringu redundantních disků nástroji které jsou součástí systému, bez nutnosti využívání aplikací třetích stran u kterých je možnost nekompatibility případně dalších komplikací.
	Mezi nevýhody se řadí větší nepřehlednost při práci se všemi disky počítače, kdy změna jednoho disky vyvolává řetězovou reakci dalších změn. Právě proto je třeba data uceleně třídit a 
	pokud možno i přehledně vizualizovat.

\subsection{LVM}
  LVM neboli logical volume management je metoda, kterou je možno spravovat diskové oddíly. Poskytuje větší flexibilitu volného místa než klasické diskové oddíly. Pracuje se třemi úrovněmi
	diskových zařízení. Prvnímy jsou fyzické svazky neboli physical volumes (PV). Fyzický svazek je tvořen buď samotným diskem, včetně například RAID disku, nebo diskovým oddílem. Fyzické
	svazky nenabízí o mnoho více funkcionality než je označení a příparava svazku pro další práci. Příprava spočívá v rozdělení fyzického svazku na fyzické extenty (physical extents, PE).

	Další úrovní jsou skupiny svazků (volume groups, VG), sdružující jak jeden nebo více fyzických svazků tak logických svazků. Skupiny svazků disponují úložným prostorem svých PV, který 
  rozdělují mezi logické svazky. Výhodou existence VG je možnost libovolně přidávat svazky a to i za plného chodu systému. Za chodu systému lze místo i ubírat ale samozřejmě
	pouze dosud neobsazenou část. 

	Třetí úrovní jsou již zmíněné logické svazky které jsou dostupné pro uživatele k ukládání jeho dat. Z tohoto pohledu se chovají se stejně jako obyčejné diskové oddíly. Jak již ale
	bylo zmíněno výhodou oproti obyčejným diskovým oddílům je flexibilita dostupného místa. Na logických svazcích je již možné vytvářet souborové systémy a dále s nimi pracovat.

	Kromě úprav velikosti za chodu je také možné přesouvat data mezi jednotlivými skupinami svazků. LVM také umí vytvářet snímky, tj. zachycovat stav dat v čase. Využítí nachází tato vlastnost
	při vytváření záloh a jako záchytný bod ke kterému je možné se vrátit. Nevýhodou logical volume managementu je skutečnost, že data na fyzických svazcích mohou být fragmentována a tak
	dochází ke snížení výkonu. Také je třeba mít na zřeteli fakt, že pokud zmenšujeme logický svazek musí tuto funkci obsahovat i souborový systém který se na něm nachází.

\subsection{LUKS}
	Linux Unified Key Setup (unifikované nastavení klíčů na linuxu) zkráceně LUKS je specifikace šifrování disků původně vytvořená pro systém Linux. Existují i implementace na jiné systémy
	těmi se zde ale zabývat nebudu. Slovy autora LUKS vznikl, aby usnadnil proces nastavování šifrovaných dat: "It has initially been developed to remedy the unpleasantness a user 
	experienced that arise from deriving the encryption setup from changing user space, and forgotten command line arguments. The result of this changes are an unaccessible encryption
	storage."\cite{on-disk-format} V současné době se používá společně s programem dm-crypt, sloužícím jako prostředek k šifrování.

	Při využití v blivetu lze šifrovat disky, diskové oddíly, logické svazky ale též i fyzické svazky. Tj. celé nastavení LVM může být šifrováno jedním klíčem, aktuálně se takto 
	standardně šifruje LVM ve Fedora linuxu pokud uživatel nastaví automatické rozdělení disku s šifrováním. Nemusí se jednat jen o pevné disky ale též o odstranitelná média jako
	CD-ROM nebo USB paměti. Též lze šifrovat odkládací prostor paměti (swap).

\subsection{Formáty souborového systému}
  Jak již bylo zmíněno blivet umí pracovat i s souborovými systémy. Struktura je následující. Výchozí je seznam zařízení reprezentující jednotlivé disky, jejich oddíly a případně
	speciální technologie jako RAID, LVM či šifrování LUKS. Každé zařízení má ale možnost mít i formát, čímž se myslí formát souborového systému. Podporována je většina známých souborových
	systémů které jsou používány dlouhou dobu. Patří mezi ně souborový systém ext, ReiserFS, XFS. Taktéž existuje podpora pro Btrfs (B-tree FS), experimentální souborový systém 
	vyvíjený společností Oracle. Přestože zatím u btrfs neexistuje stabilní verze je mezi distribucemi podporován, neboť nabízí řešení některých problémů současných filesystémů.

\section{Knihovna Graphviz}
	Graphviz je program sloužící k vizualizaci dat formou grafů, ve smyslu orientovaných či neorientovaných. Pomocí něj je možné generovat grafy sloužící k znázornění počítačové sítě nebo
	vztahů mezi určitými objekty. Nelze vytvářet grafy průběhů funkcí či grafy znázorňující vztahy mezi číselnými hodnotami. Jinými slovy graphviz generuje grafy které známe z teorie grafů,
	ale není schopen generovat grafy známé například z ekonomie.

	Program je z velké části napsán v jazyce C ale ve všech v současnosti používaných jazycích pro něj existují obalovací (wrapper) knihovny. Z těchto knihoven se budeme soustředit hlavně na
	knihovnu pro Python 3 která je používána v mé práci. Pokud bychom graphviz spouštěli jako program a pracovali s ním přímo například z příkazové řádky, využívali bychom jeho vlastní jazyk
	na definici grafů nazvaný DOT. Tento jazyk je definovaný následovně:

	"The following is an abstract grammar defining the DOT language. Terminals are shown in bold font and nonterminals in italics. Literal characters are given in single quotes. Parentheses
	( and ) indicate grouping when needed. Square brackets [ and ] enclose optional items. Vertical bars | separate alternatives.
	graph			:		[ \textbf{strict} ] (\textbf{graph} | \textbf{digraph}) [ ID ] \textbf{\'\{\'} stmt_list \textbf{\'\}\'}
	stmt_list :		[ stmt [ \textbf{';'} ] stmt_list ]
	stmt			:		node_stmt
	|							edge_stmt
	|							attr_stmt
	|							ID '=' ID
	|							subgraph
	attr_stmt	:	(\textbf{graph} | \textbf{node} | \textbf{edge}) attr_list
	attr_list	:	'[' [ a_list ] ']' [ attr_list ]
	a_list  	:	ID '=' ID [ (';' | ',') ] [ a_list ]
	edge_stmt	:	(node_id | subgraph) edgeRHS [ attr_list ]
	edgeRHS		:	edgeop (node_id | subgraph) [ edgeRHS ]
	node_stmt	:	node_id [ attr_list ]
	node_id		:	ID [ port ]
	port			:	':' ID [ ':' compass_pt ]
	|						':' compass_pt
	subgraph	:	[ \textbf{subgraph} [ ID ] ] \textbf{'\{'} stmt_list \textbf{'\}'}
	compass_pt:	(\textbf{n}| \textbf{ne} | \textbf{e} | \textbf{se} | \textbf{s} | \textbf{sw} | \textbf{w}  | \textbf{nw} | \textbf{c} | \textbf{_})"

	Vidíme že hlavnímy elementy při vytváření grafu jsou graph (graf), uzel (node) a hrana (edge). Základní grafy lze tvořit jen s pomocí těchto tří klíčových slov. Nyní je rozeberu detailněji.

\subsection{Graf}
	Klíčové slovo graph uvádí jakýkoliv graf který bude vytvořen. Včetně prvního kořenového grafu. Každý zápis v jazyce DOT tedy musí začínat slovem graph nebo digraph s výjimkou užití slova
	strict. Pokud jej použijeme zamezíme vzniku vícenásobných hran, neboli mezi každým počátečním a koncovým uzlem bude maximálně jedna hrana. 

	Rozdíly mezi slovy graph a digraph jsou zřejmé. Graph označuje graf neorientovaný ve kterém jsou hrany bez šipek na koncích. Slovo digraph je zkrácené spojení directed graph (orientovaný
	graf) a vyjadřuje tedy graf orientovaný ve kterém na koncích hran jsou šipky. Při vizualizaci rozdělení diskového prostoru jsem se rozhodl použít právě orientované grafu neboť považuji za 
	důležité poskytnout uživateli co nejvíce vodítek ke znázornění vztahu rodič -> potomek a orientované hrany jsou ideálním případem pro tuto pomoc. 

	Speciálním případem grafu je i podgraf (subgraph). Grafy mohou být vkládány do sebe. S pomocí podgrafů lze snížit velikost zdrojového kódu v jazyce DOT. Příkladem budiž situace kdy zápisem
	hrany vedoucí od uzlu k podgrafu ( A -- \{B C\}) vytvoříme stejný efekt jako při zápisu každé jednotlivé dvojice mezi uzlem A a uzly B a C (A -- B, A -- C). Dále je možné podgrafy využít
	pro specifikaci odlišných atributů uzlů či hran. V podgrafu lze jednoduše nastavit jiné atributy než ve zbytku grafu.

	Poslední situací pro využití podgrafu je situace kdy chceme uzly shlukovat. V tomto případě je nutné přidat klíčové slovo cluster k názvu podgrafu a určité grafovací stroje tyto uzly 
	napozicují k sobě do skupiny.

\subsection{Uzly}
	Uzly jsou základem každého grafu. V graphvizu jsou definovány svým jménem. Základní graf s jedním uzlem definujeme v jazyce DOT jednoduše a to pomocí: graf { A }. Tento zápis vytvoří
	graf s jedním uzlem uprostřed v jehož středu bude napsán název uzlu, v našem případě "A". Uzly mají mnoho různých atributů od svého jména po url odkazy a atributy HTML formátování. 
	Nejdůležitějšímy ale jsou ty které jsou definovány v základním nastavení. Jsou to tvar uzlu (shape), barva výplně (fillcolor) a jméno (name) nebo nálepka (label). 
	
	Standardně je tvar nastaven na elipsu a barva
	na bílou. Jméno se bere podle identifikátory při vytváření uzlu. Nálepka je rozšířením jména. Pokud je definována jméno nahrazuje a její obsah je vepsán dovnitř uzlu. Od jména se liší právě
	tím že její obsah lze libovolně upravovat, avšak do ní zle ukládat pouze text, jakékoliv speciální formátování nebude zohledněno. Pro vkládání obrázků složí atribut image a pro vkládání
	odkazů atribut url.

	Právě třemi základnímy atributy jsem se rozhodl rozlišovat jednotlivé typy zařízení se kterými pracuje blivet. S pomocí tvaru uzlu odlišuji technologie zařízení. Čím je technologie
	abstraktnější tím zaoblenější tvar má. Pevné disky a jejich oddíly jsou znázorněny čtverci, LVM čtverci se zaoblenými rohy a disky připojené přes síť či šifrování je znázorněno elipsou.
	Pro toto dělení jsem se rozhold abych zachoval konzistenci celého grafu a naváděl uživatele ke schopnosti rozlišit rozdíly jen s pomocí krátkého shlédnutí tvaru.

	Druhým odlišovacím prvkem je barva. Jak její odstím tak sytost. Některé odstíny jsou rezervovány pro odlišení akcí které budou provedeny při konfiguraci. Zelenou barvou jsou vyznačeny
	nově se objevivší prvky, červenou zaniknuvší prvky a oranžovou prvky u kterých došlo ke změnám. Sytost barvy společně s tvarem jasně definují použitou technologii. Barva pomáhá tam kde
	samotný tvar nestačí. Čili pokud používá jak šifrování tak disk připojený přes internet stejný tvar, tak poté k jejich jednoznačnému odlišení dojde použitím sytější a méně syté barvy.

\subsection{Hrany}
	Hrany nejsou tak komplikovanými elementy jako uzly. Jejich základnímy atributy jsou počáteční a koncový uzel. U neorientovaného grafu nelze rozeznat který uzel je počáteční. Hrany 
	nepoužívají tolik atributů jako uzly avšak například nálepku stále mít mohou. Vzhledem k jejich tvaru je u nich zbytečný atribut výplňové barvy (fillcolor) a používá se jen barva "pera".
	Jak již bylo zmíněno v mé praci jsou všechny grafy orientované a tak záleží na pořadí v jakém jsou uzly předány funkci která tvoří hrany. Nicméně směr je vždy od rodiče k potomkovi a 
	nikdy naopak.
	
\subsection{Rozložení}

	\printbibliography

\end{document}
