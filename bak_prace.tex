\documentclass[color,table,oneside,nolot,nolof]{fithesis}
\usepackage[resetfonts]{cmap}
\usepackage[main=czech,english]{babel}
\usepackage[utf8]{inputenc}
\thesissetup{
		university = mu,
		faculty = fi,
		type = bc,
		author = Václav Hodina,
		gender = m,
		advisor = Marek Grác,
		title = {Vizualizace rozdělování disků},
		keywords = {vizualizace, instalace, rozdělení disků, linux, LVM, RAID},
}
\thesislong{abstract}{
	Práce popisuje vývoj modulu do instalátoru Anaconda, který využívají linuxové distribuce vycházející z~Red Hat Enterprise Linuxu.
}
\thesislong{thanks}{
	Zde chci poděkovat Marku Grácovi za vedení mé práce, Marii Staré za korektury.
}
\usepackage{csquotes}
\usepackage[plainpages=false,pdfpagelabels,unicode]{hyperref}
\usepackage{charter,graphicx}
\usepackage[top=1in, bottom=1.25in, left=1.25in, right=1.25in]{geometry}
\usepackage[
	backend=biber,
	style=numeric,
	citestyle=numeric-comp,
	sorting=none,
	sortlocale=auto
	]{biblatex}
\addbibresource{bak_prace.bib}
\usepackage{makeidx}
\makeindex
\usepackage{paralist} 
\usepackage{amsmath} 
\usepackage{amsthm} 
\usepackage{amsfonts} 
\usepackage{url} 
\usepackage{menukeys}
\hyphenation{graph-viz}
\begin{document}
\chapter{Úvod}
	Bakalářská práce zpracovává řešení problémů s~vizualizací rozdělených disků při instalaci systému. Cílem práce je vytvořit pochopitelnou grafickou nápovědu pro administrátory 
	počítačů, zvláště serverů s~mnoha disky. 
	Konkrétněji se jedná o~rozšíření instalátoru Anaconda, které zpracovává informace o~jednotlivých discích, jako je název, velikost a typ disku a~jednotlivé oddíly na disku utvořené. 
	Samozřejmostí je zahrnutí 
	diskových polí typu RAID (Redundant Array of Independent Disks)  a~virtualizovaných disků mezi vizualizovaná data; rozšíření počítá se všemi těmito typy. Data jsou uložena 
	ve vlastních třídách tak, aby programování případné další funcionality 
	nepředstavovalo problém. Program vytváří graf podobný stromové struktuře a~zobrazuje jej uživateli. Graf se během instalace tvoří dvakrát, poprvé
	před rozdělením disků a~podruhé pro kontrolu, zda jsou předložené změny korektní, než se zformátují disky. V~současnosti je k~tomuto účelu využíván pouze textový 
	seznam změn, který je nedostatečný. Člověk dokáže mnohem lépe a~rychleji kontrolovat obrázková data než homogenní text. 

	Práce vznikala nejen na Fakultě informatiky Masarykovy univerzity (FI MUNI), ale i~ve společnosti Red Hat. Tam budou využity její výsledky,
	integrované do instalátoru Anaconda, který je v~současnosti používán v~linuxových distribucích Red Hat Enterprise Linux (RHEL), CentOs a~Fedora.

	Práci jsem si vybral z~několika důvodů.  Možnost podílet se na vývoji svobodného softwaru je pro mě velmi důležitým hlediskem 
	při psaní jakéhokoliv programu. Druhý důvod je možný rozsah uplatnitelnosti výsledků mé práce. Každý systém je třeba nejprve nainstalovat, výsledky
	této práce tedy uvidí velké množství lidí, což je bezpochyby velká motivace pro každého, kdo něco tvoří. Třetím důvodem je výběr programovacího jazyka, který je vyžadován 
	v~zadání. Jedná se o jazyk Python, který považuji za velmi flexibilní, aniž by byly kladeny přílišné nároky na výkon systému. 

	Jak jsem zmínil výše, hlavním cílem práce je naprogramování aplikace, která vytváří graf stromové struktury rozdělených disků. Data přijímá od instalátoru Anaconda, 
	přičemž využívá knihovnu blivet. 
	Vedlejšími cíli je možnost vytvářet grafy ze souborů XML, ne jen z~aplikace, dále integrace do instalátoru s~možností označovat jednotlivé diskové oddíly a~interagovat
	s~nimi. Posledním doplňkovým cílem je funkcionalita umožňující porovnat stav před instalací a~po ní v~případě, že je systém přeinstalováván.

	Z~cílů vychází také struktura práce. První kapitola popisuje použité knihovny. První knihovna, blivet\cite{blivet}, poskytuje data a mimo jiné může sloužit i~pro změny 
	nastavení disků (tato funkcionalita je v~mé práci zmíněna jen okrajově). Druhou je knihovna graphviz-python\cite{graphviz-python}, Graphviz je program pro tvorbu grafů. Kromě
	jednoduchého spojování uzlů hranami dokáže uzly automaticky třídit a~logicky rozmisťovat podle různých přednastavených pravidel. Nabízí také různý vzhled uzlů
	a~hran a výsledné grafy dokáže exportovat v~několika formátech. Výsledek mé práce operuje především s~formátem škálovatelné vektorové grafiky (SVG). Graphviz-python je 
	nadstavba Graphvizu pro použití v~jazyce Python.

	Druhá kapitola je o~mém návrhu jednotlivých tříd programu, jejich dokumentaci a~popisu funkcí. Nejdůležitější z tříd jsou ty pro uzly a~hrany. Zmíněny jsou také pomocné třídy 
	pro načítání z~jiných formátů vstupních dat, jako je již zmiňované XML.

	Třetí kapitola obdobně popisuje návrh vzhledu aplikace a~její chování. Zdůvodňuje, proč jsem se rozhodl pro jednotlivé grafické prvky a~barevná odlišení.

	Čtvrtá kapitola obsahuje ukázky práce programu. Demonstruje několik konfigurací, jež mají za úkol program otestovat a~vyzkoušet i~potenciálně problémové situace. Ukázky  
	zahrnují situace jak při práci v~prostředí instalátoru, tak mimo něj.

	Pátá kapitola zmiňuje další možná rozšíření mého programu. 

\chapter{Stať}
\section{Knihovna Blivet}
	První a nejdůležitější knihovouk která je v mé práci využívána je knihovna blivet. Jde o projekt a bakalářsknou práci Bc. Vojtěcha Trefného. Projekt vznikl, stejně jako moje
	práce ve firmě Red Hat a slouží k rozšíření již zmiňovaného instalátoru Anaconda. Použití této knihovny je součástí zadání a proto nebudu diskutovat její výhody a nevýhody
	oproti ostatním knihovnám. 

	Blivet však není knihovnou která by jen četla informace o pevných discích. Mezi její funkce patří i konfigurace různých datových úložišť. Nemusí se ani jednat jen o pevné disky.
	Ovládá i mnohé další technologie se kterými se lze v dnešní době setkat. Příkladem jsou to vícenásobné pole nezávislých disků (RAID), technologie logických svazků disku (LVM) či 
	ovládání zašifrovaných modulů pomocí technologie LUKS. Všechny 3 příklady proberu níže.

	Dále obsahuje též nástroje pro práci se souborovými systémy diskových oddílů od starších a již používaných jako jsou žurnálovací souborové systémy ext2 až ex4 či ReiserFS po novější
	jakými jsou btrfs nebo ZFS. Též se stará o bootovací oddíly čili master boot record (MBR) a GUID partition table (GPT). Stručně řečeno stará se o všechny součásti procesu instalace
	nového systému na počítač.

\subsection{RAID}
	Vícenásobná pole nezávislých disků jsou velmi elegantní ochranou před selháním disků. Existůjí různé způsoby jak pole realizovat ale základní princip zůstává vždy stejný. Jde v~něm
	o~několik disků, které nakonec vystupují jako jeden disk. Podle použítí typu vícenásobného diskového pole může mít tento disk kapacitu rovnou součtu disků jej tvořících, anebo také
	jen kapacitu jednoho disku, přičemž data jsou zrcadlena na ostatní disky. Cílem tohoto nastavení je ochrana před selháním hardware a ztrátou dat. 
	
	Blivet obsahuje nástroje pro práci s svobodnou technologií mdadm která slouží k nastavení softwarového RAID pole. Při využití hardwarových technologií, zvláště pak nesvobodných
	je možné k tomuto RAIDu přistupovat jako k obyčejnému disku, čehož i nesvobodné RAIDy často využívájí a svých mnoho disků schovávají za jednotným rozhraním, které se tváří jako jeden
	disk. 

	Program mdadm je ale softwarový RAID a proto má počítač celou dobu přehled nejen o finálním zabezpečeném disku proti selhání ale i o všech dílčích discích které ho tvoří. Výhodou je
	možnost monitoringu redundantních disků nástroji které jsou součástí systému, bez nutnosti využívání aplikací třetích stran u kterých je možnost nekompatibility případně dalších komplikací.
	Mezi nevýhody se řadí větší nepřehlednost při práci se všemi disky počítače, kdy změna jednoho disky vyvolává řetězovou reakci dalších změn. Právě proto je třeba data uceleně třídit a 
	pokud možno i přehledně vizualizovat.

\subsection{LVM}
  LVM neboli logical volume management je metoda, kterou je možno spravovat diskové oddíly. Poskytuje větší flexibilitu volného místa než klasické diskové oddíly. Pracuje se třemi úrovněmi
	diskových zařízení. Prvnímy jsou fyzické svazky neboli physical volumes (PV). Fyzický svazek je tvořen buď samotným diskem, včetně například RAID disku, nebo diskovým oddílem. Fyzické
	svazky nenabízí o mnoho více funkcionality než je označení a příparava svazku pro další práci. Příprava spočívá v rozdělení fyzického svazku na fyzické extenty (physical extents, PE).

	Další úrovní jsou skupiny svazků (volume groups, VG), sdružující jak jeden nebo více fyzických svazků tak logických svazků. Skupiny svazků disponují úložným prostorem svých PV, který 
  rozdělují mezi logické svazky. Výhodou existence VG je možnost libovolně přidávat svazky a to i za plného chodu systému. Za chodu systému lze místo i ubírat ale samozřejmě
	pouze dosud neobsazenou část. 

	Třetí úrovní jsou již zmíněné logické svazky které jsou dostupné pro uživatele k ukládání jeho dat. Z tohoto pohledu se chovají se stejně jako obyčejné diskové oddíly. Jak již ale
	bylo zmíněno výhodou oproti obyčejným diskovým oddílům je flexibilita dostupného místa. Na logických svazcích je již možné vytvářet souborové systémy a dále s nimi pracovat.

	Kromě úprav velikosti za chodu je také možné přesouvat data mezi jednotlivými skupinami svazků. LVM také umí vytvářet snímky, tj. zachycovat stav dat v čase. Využítí nachází tato vlastnost
	při vytváření záloh a jako záchytný bod ke kterému je možné se vrátit. Nevýhodou logical volume managementu je skutečnost, že data na fyzických svazcích mohou být fragmentována a tak
	dochází ke snížení výkonu. Také je třeba mít na zřeteli fakt, že pokud zmenšujeme logický svazek musí tuto funkci obsahovat i souborový systém který se na něm nachází.

\subsection{LUKS}
	Linux Unified Key Setup (unifikované nastavení klíčů na linuxu) zkráceně LUKS je specifikace šifrování disků původně vytvořená pro systém Linux. Existují i implementace na jiné systémy
	těmi se zde ale zabývat nebudu. Slovy autora LUKS vznikl, aby usnadnil proces nastavování šifrovaných dat: "It has initially been developed to remedy the unpleasantness a user 
	experienced that arise from deriving the encryption setup from changing user space, and forgotten command line arguments. The result of this changes are an unaccessible encryption
	storage."\cite{on-disk-format} V současné době se používá společně s programem dm-crypt, sloužícím jako prostředek k šifrování.

	Při využití v blivetu lze šifrovat disky, diskové oddíly, logické svazky ale též i fyzické svazky. Tj. celé nastavení LVM může být šifrováno jedním klíčem, aktuálně se takto 
	standardně šifruje LVM ve Fedora linuxu pokud uživatel nastaví automatické rozdělení disku s šifrováním. Nemusí se jednat jen o pevné disky ale též o odstranitelná média jako
	CD-ROM nebo USB paměti. Též lze šifrovat odkládací prostor paměti (swap).

\section{Formáty souborového systému}
















	\printbibliography

\end{document}
