\documentclass[color,table,oneside,nolot,nolof]{fithesis}
\usepackage[resetfonts]{cmap}
\usepackage[main=czech,english]{babel}
\usepackage[utf8]{inputenc}
\thesissetup{
		university = mu,
		faculty = fi,
		type = bc,
		author = Václav Hodina,
		gender = m,
		advisor = Marek Grác,
		title = {Nástroj pro konfiguraci datových úložišť OS GNU/Linux},
		keywords = {vizualizace, instalace, rozdělení disků, linux, LVM},
}
\thesislong{abstract}{
	Práce popisuje vývoj modulu do instalátoru Anaconda, který využívají linuxové distribuce vycházející z~Red Hat Enterprise Linuxu.
}
\thesislong{thanks}{
	Zde chci poděkovat Marku Grácovi za vedení mé práce, Marii Staré za korektury a~své rodinně za podporu.
}
\usepackage{csquotes}
\usepackage[plainpages=false,pdfpagelabels,unicode]{hyperref}
\usepackage{charter,graphicx}
\usepackage[top=1in, bottom=1.25in, left=1.25in, right=1.25in]{geometry}
\usepackage[
	backend=biber,
	style=numeric,
	citestyle=numeric-comp,
	sorting=none,
	sortlocale=auto
	]{biblatex}
\addbibresource{bak_prace.bib}
\usepackage{makeidx}
\makeindex
\usepackage{paralist} 
\usepackage{amsmath} 
\usepackage{amsthm} 
\usepackage{amsfonts} 
\usepackage{url} 
\usepackage{menukeys}
\hyphenation{graph-viz}
\begin{document}
\chapter{Úvod}
	Bakalářská práce zpracovává řešení problémů s~vizualizací rozdělených disků při instalaci systému. Cílem práce je vytvořit pochopitelnou grafickou nápovědu pro administrátory 
	počítačů, zvláště serverů s~mnoha disky. 
	Konkrétněji se jedná o~rozšíření instalátoru Anaconda, které zpracovává informace o~jednotlivých discích, jako je název, velikost a typ disku a~jednotlivé oddíly na disku utvořené. 
	Samozřejmostí je zahrnutí 
	diskových polí typu RAID (Redundant Array of Independent Disks)  a~virtualizovaných disků mezi vizualizovaná data; rozšíření počítá se všemi těmito typy. Data jsou uložena 
	ve vlastních třídách tak, aby programování případné další funcionality 
	nepředstavovalo problém. Program vytváří graf podobný stromové struktuře a~zobrazuje jej uživateli. Graf se během instalace tvoří dvakrát, poprvé
	před rozdělením disků a~podruhé pro kontrolu, zda jsou předložené změny korektní, než se zformátují disky. V~současnosti je k~tomuto účelu využíván pouze textový 
	seznam změn, který je nedostatečný. Člověk dokáže mnohem lépe a~rychleji kontrolovat obrázková data než homogenní text. 

	Práce vznikala nejen na Fakultě informatiky Masarykovy univerzity (FI MUNI), ale i~ve společnosti Red Hat. Tam budou využity její výsledky,
	integrované do instalátoru Anaconda, který je v~současnosti používán v~linuxových distribucích Red Hat Enterprise Linux (RHEL), CentOs a~Fedora.

	Práci jsem si vybral z~několika důvodů.  Možnost podílet se na vývoji svobodného softwaru je pro mě velmi důležitým hlediskem 
	při psaní jakéhokoliv programu. Druhý důvod je možný rozsah uplatnitelnosti výsledků mé práce. Každý systém je třeba nejprve nainstalovat, výsledky
	této práce tedy uvidí velké množství lidí, což je bezpochyby velká motivace pro každého, kdo něco tvoří. Třetím důvodem je výběr programovacího jazyka, který je vyžadován 
	v~zadání. Jedná se o jazyk Python, který považuji za velmi flexibilní, aniž by byly kladeny přílišné nároky na výkon systému. 

	Jak jsem zmínil výše, hlavním cílem práce je naprogramování aplikace, která vytváří graf stromové struktury rozdělených disků. Data přijímá od instalátoru Anaconda, 
	přičemž využívá knihovnu blivet. 
	Vedlejšími cíli je možnost vytvářet grafy ze souborů XML, ne jen z~aplikace, dále integrace do instalátoru s~možností označovat jednotlivé diskové oddíly a~interagovat
	s~nimi. Posledním doplňkovým cílem je funkcionalita umožňující porovnat stav před instalací a~po ní v~případě, že je systém přeinstalováván.

	Z~cílů vychází také struktura práce. První kapitola popisuje použité knihovny. První knihovna, blivet\cite{blivet}, poskytuje data a mimo jiné může sloužit i~pro změny 
	nastavení disků (tato funkcionalita je v~mé práci zmíněna jen okrajově). Druhou je knihovna graphviz-python\cite{graphviz-python}, Graphviz je program pro tvorbu grafů. Kromě
	jednoduchého spojování uzlů hranami dokáže uzly automaticky třídit a~logicky rozmisťovat podle různých přednastavených pravidel. Nabízí také různý vzhled uzlů
	a~hran a výsledné grafy dokáže exportovat v~několika formátech. Výsledek mé práce operuje především s~formátem škálovatelné vektorové grafiky (SVG). Graphviz-python je 
	nadstavba Graphvizu pro použití v~jazyce Python.

	Druhá kapitola je o~mém návrhu jednotlivých tříd programu, jejich dokumentaci a~popisu funkcí. Nejdůležitější z tříd jsou ty pro uzly a~hrany. Zmíněny jsou také pomocné třídy 
	pro načítání z~jiných formátů vstupních dat, jako je již zmiňované XML.

	Třetí kapitola obdobně popisuje návrh vzhledu aplikace a~její chování. Zdůvodňuje, proč jsem se rozhodl pro jednotlivé grafické prvky a~barevná odlišení.

	Čtvrtá kapitola obsahuje ukázky práce programu. Demonstruje několik konfigurací, jež mají za úkol program otestovat a~vyzkoušet i~potenciálně problémové situace. Ukázky  
	zahrnují situace jak při práci v~prostředí instalátoru, tak mimo něj.

	Pátá kapitola zmiňuje další možná rozšíření mého programu. 

\chapter{Stať}
\section{Knihovna Blivet}
	První a nejdůležitější knihovouk která je v mé práci využívána je knihovna blivet. Jde o projekt a bakalářsknou práci Bc. Vojtěcha Trefného. Projekt vznikl, stejně jako moje
	práce ve firmě Red Hat a slouží k rozšíření již zmiňovaného instalátoru Anaconda. Použití této knihovny je součástí zadání a proto nebudu diskutovat její výhody a nevýhody
	oproti ostatním knihovnám. 

	Blivet však není knihovnou která by jen četla informace o pevných discích. Mezi její funkce patří i konfigurace různých datových úložišť. Nemusí se ani jednat jen o pevné disky.
	Ovládá i mnohé další technologie se kterými se lze v dnešní době setkat. Příkladem jsou to vícenásobné pole nezávislých disků (RAID), technologie logických svazků disku (LVM) či 
	ovládání zašifrovaných modulů pomocí technologie LUKS. Všechny 3 příklady proberu níže.

\subsection{RAID}
	Vícenásobná pole nezávislých disků jsou velmi elegantní ochranou před selháním disků. Existůjí různé způsoby jak pole realizovat ale základní princip zůstává vždy stejný. Jde v~něm
	o~několik disků, které nakonec vystupují jako jeden disk. Podle použítí typu vícenásobného diskového pole může mít tento disk kapacitu rovnou součtu disků jej tvořících, anebo také
	jen kapacitu jednoho disku, přičemž data jsou zrcadlena na ostatní disky.

\subsection{LVM}

\subsection{LUKS}

















	\printbibliography

\end{document}
